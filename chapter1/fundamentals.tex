%%%%%%%%%%%%%%%%%%%%%%%%%%%%%%%%%%%%%%%%%
% Author: Sibi <sibi@psibi.in>
%%%%%%%%%%%%%%%%%%%%%%%%%%%%%%%%%%%%%%%%%
\documentclass{article}
\usepackage{graphicx}
\usepackage{verbatim}
\usepackage{amsmath}
\usepackage{amsfonts}
\usepackage{amssymb}
\usepackage{tabularx}
\usepackage{mathtools}
\newcommand{\BigO}[1]{\ensuremath{\operatorname{O}\bigl(#1\bigr)}}
\setlength\parskip{\baselineskip}
\begin{document}
\title{Chapter 1 - Fundamentals}
\author{Sibi}
\date{\today}
\maketitle

% See here: http://tex.stackexchange.com/a/43009/69223
\DeclarePairedDelimiter\abs{\lvert}{\rvert}%
\DeclarePairedDelimiter\norm{\lVert}{\rVert}%

% Swap the definition of \abs* and \norm*, so that \abs
% and \norm resizes the size of the brackets, and the 
% starred version does not.
\makeatletter
\let\oldabs\abs
\def\abs{\@ifstar{\oldabs}{\oldabs*}}
%
\let\oldnorm\norm
\def\norm{\@ifstar{\oldnorm}{\oldnorm*}}
\makeatother
\newpage

\section{Solution 1}
\begin{itemize}
\item $\{ \epsilon, a, b, c, aa, bb, cc, ab, ba, ac, ca, bc, cb \}$
\item $\{ \epsilon, aaa,aaaa,aaaa,aaaaa \}$
\item $\{ \epsilon \}$
\item $\{ \epsilon, 0,3,6,9,12,15,18 \}$
\item $\{ \epsilon, "b","bb","a","ab","abb" \}$
\end{itemize}

\section{Solution 2}
$\{a^nb^n\}$ \\
$\{a^nb^m\ | n > m\}$

\end{document}
